\chapter*{摘\hspace{1em}要}
\chabstractmark

\vspace{4mm}
为进一步规范我校研究生学位论文撰写格式,提高研究生学位论文质量,参照国家标准《学术论文编写规则》(GB/T 7713.2-2022),结合我校实际,制定本范式。

相对于上一版的主要变化如下:

1. 按照最新范式重新编写文档,预置主要格式样式,可作为论文模板使用。

2. 更新摘要、绪论、正文、结论撰写说明,以及全文语言、表述注意事项。

3. 参照最新国家标准,调整参考文献格式要求。

4. 明确部分格式要求细节,如表格的样式,图表附注的格式,图表跨页等。

5. 附录增加西南交通大学研究生培养二级单位、常见一级学科中英文对照表以及西南交通大学专业学位类别。

此文档可在研究生院网站下载,如有变动,以研究生院网站最新公布的版本为准。

\par
\vspace{1.5em}
\noindent\textbf{关键词:} 学位论文,撰写格式,主要变化

\chapter*{\textbf{ABSTRACT}}
\enabstractmark

\vspace{5mm}
% 禁止连字符
\hyphenpenalty=10000
\exhyphenpenalty=10000
\tolerance=1000
In order to further standardize the format of dissertation/thesis writing and improve graduate dissertation/thesis quality, this specification is formulated with reference to the national standard "Presentation of academic papers" (GB/T 7713.2-2022) and the reality of SWJTU.

The main changes in this revision from the last version are as follows.

1. Rewrite the document according to the requirements of the specifications, and preset main formatting styles, which can be used as a template directly.

2. Update the writing instructions of abstract, introduction, main chapters, and conclusions, as well as the notes on language and presentation.

3. Adjust the format requirements of references according to the latest national standards.

4. Clarify some details of format requirements, e.g., three-line style for tables, format for figure/table annotations, the problem of cross page figure/table.

5. Add comparison tables of Chinese and English names of colleges, academic degree disciplines, and professional degree categories in Appendixes.

This document can be downloaded from the Graduate School website. In case of any changes, the latest version published on the Graduate School website shall prevail.

\par
\vspace{1.5em}
\noindent\textbf{Keywords:} Dissertation/Thesis, Writing Format, Main changes 